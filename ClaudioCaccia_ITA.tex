%%%%%%%%%%%%%%%%%%%%%%%%%%%%%%%%%%%%%%%%%
% "ModernCV" CV and Cover Letter
% LaTeX Template
% Version 1.11 (19/6/14)
%
% This template has been downloaded from:
% http://www.LaTeXTemplates.com
%
% Original author:
% Xavier Danaux (xdanaux@gmail.com)
%
% License:
% CC BY-NC-SA 3.0 (http://creativecommons.org/licenses/by-nc-sa/3.0/)
%
% Important note:
% This template requires the moderncv.cls and .sty files to be in the same 
% directory as this .tex file. These files provide the resume style and themes 
% used for structuring the document.
%
%%%%%%%%%%%%%%%%%%%%%%%%%%%%%%%%%%%%%%%%%

%----------------------------------------------------------------------------------------
%	PACKAGES AND OTHER DOCUMENT CONFIGURATIONS
%----------------------------------------------------------------------------------------

\documentclass[11pt,a4paper,sans]{moderncv} % Font sizes: 10, 11, or 12; paper sizes: a4paper, letterpaper, a5paper, legalpaper, executivepaper or landscape; font families: sans or roman

\moderncvstyle{casual} % CV theme - options include: 'casual' (default), 'classic', 'oldstyle' and 'banking'
\moderncvcolor{blue} % CV color - options include: 'blue' (default), 'orange', 'green', 'red', 'purple', 'grey' and 'black'

\usepackage{lipsum} % Used for inserting dummy 'Lorem ipsum' text into the template

\usepackage[scale=0.75]{geometry} % Reduce document margins
%\setlength{\hintscolumnwidth}{3cm} % Uncomment to change the width of the dates column
%\setlength{\makecvtitlenamewidth}{10cm} % For the 'classic' style, uncomment to adjust the width of the space allocated to your name

%----------------------------------------------------------------------------------------
%	NAME AND CONTACT INFORMATION SECTION
%----------------------------------------------------------------------------------------

\firstname{Claudio} % Your first name
\familyname{Caccia} % Your last name

% All information in this block is optional, comment out any lines you don't need
\title{Curriculum Vitae}
\address{Busto Arsizio, 21052 VA}{via Gramsci 4}
\mobile{(+39) 329 3738162}
\phone{Skype: ccaccia.atos.com}
%\fax{(000) 111 1113}
\email{c.caccia@libero.it}
%\homepage{ccaccia73.github.io/}{ccaccia73.github.io} % The first argument is the url for the clickable link, the second argument is the url displayed in the template - this allows special characters to be displayed such as the tilde in this example
\extrainfo{Autorizzo il trattamento dei miei dati personali, ai sensi del D.lgs. 196/03}
%\photo[60pt][0.2pt]{pictures/picture} % The first bracket is the picture height, the second is the thickness of the frame around the picture (0pt for no frame)
%\quote{"All models are wrong, but some are useful" George E.P. Box}

%----------------------------------------------------------------------------------------

\begin{document}

\makecvtitle % Print the CV title


%----------------------------------------------------------------------------------------
%	WORK EXPERIENCE SECTION
%----------------------------------------------------------------------------------------

\section{Esperienze Lavorative}

\subsection{Principali}

\cventry{2012--attuale}{R\&D Engineer}{Atos S.p.A.}{Sesto Calende}{impiegato}{Progettazione, dimensionamento e verifica di elettrovalvole oleodinamiche, ottimizzazione delle performance mediante strumenti di \emph{Virtual Prototyping}.\newline{}
	In particolare:
	\begin{itemize}
		\item Simulazione fluidodinamica (\emph{CFD}) per:
		\begin{itemize}
			\item stimare ed ottimizzare le \emph{performance} di componenti nuovi ed esistenti
			\item determinare le forze che agiscono su elementi mobili
		\end{itemize}
		\item Calcolo strutturale (\emph{FEM}) per:
		\begin{itemize}
			\item determinare sforzi e deformazioni di componenti
			\item stabilire ed ottimizzare le sezioni critiche
			\item analizzare le frequenze proprie e la risposta in frequenza
		\end{itemize}
		\item Utilizzo di tecniche di \emph{DoE} per la ricerca delle condizioni di funzionamento ottimali minimizzando il numero di prove o simulazioni
		\item Testing a banco (supervisione ed esecuzione) per la verifica delle simulazioni effettuate
		\item Simulazione di sistemi idraulici per la definizione delle performance e
		determinazione della componentistica pi\`u adatta alle esigenze del cliente
		\item determinazione dei protocolli di test e redazione della documentazione tecnica
	\end{itemize}
}

\cventry{2008--2012}{Firmware Engineer}{Atos S.p.A.}{Sesto Calende}{impiegato}{Sviluppo Firmware di attuatori elettroidraulici per il controllo di portata, di pressione, di forza e di posizione.
In particolare:
\begin{itemize}
	\item Implementazione di algoritmi di controllo per movimentazione assi oleodinamici definendo:
	\begin{itemize}
		\item modi di funzionamento
		\item logiche di transizione
		\item sequenze automatiche programmabili di lavoro
		\item profili di velocit\`a
	\end{itemize}
	\item \emph{Testing} a banco dei prototipi realizzati
	\item Ottimizzazione dei parametri di controllo dei componenti da immettere in produzione
\end{itemize}
}


\cventry{2007--2008}{Assegnista di Ricerca}{Universit\`a di Milano Bicocca}{Dipartimento di Informatica}{}
{Collaborazione con il team di ricerca in \emph{Intelligenza Artificiale} e \emph{Robotica} con i seguenti compiti:
\begin{itemize}
	\item Partecipazione al progetto di ricerca \emph{Rawseeds} (\url{www.rawseeds.org}). Obiettivo del
	progetto: pubblicazione di dati utili allo sviluppo ed alla verifica di algoritmi di SLAM
	(Simultaneous Localization and Mapping). Attivit\`a svolte:
	\begin{itemize}
		\item progettazione e realizzazione di piattaforme robotiche
		\item equipaggiamento con sensori
		\item esecuzione delle campagne di acquisizione
	\end{itemize}
	\item Sviluppo di robot per il Progetto \emph{Robocup} (\url{www.robocup.org}): progettazione di componenti e delle
	piattaforme
\end{itemize} }


\cventry{2004--2007}{Proposal Engineer}{Siemens}{Milano}{collaboratore a progetto}{
Progettazione e dimensionamento di sistemi informativi, di gestione ed
archiviazione dati nel campo medicale, in particolare per il trattamento e la
conservazione di dati ed immagini provenienti da apparecchiature diagnostiche digitali (\textsc{tac, rm} \dots)}

\cventry{2000--2003}{Collaboratore alla Ricerca}{Politecnico di Milano}{sede di Como}{}
{Attivit\`{a} di didattica, ricerca e consulenza per il Dipartimento di
	Ingegneria Gestionale nella sede di Como, in particolare:
	\begin{itemize}
		\item collaborazioni con aziende tessili del territorio comasco finalizzate alla caratterizzazione ed al miglioramento della qualit\`{a} dei tessuti
		\item didattica presso il polo di Como, svolgendo esercitazioni relative a corsi di
		\emph{Impianti Tessili} e \emph{Gestione della Produzione Industriale}
		\item permanenza presso l'\emph{ETH} di Zurigo e collaborazione allo sviluppo di una macchina automatica per la realizzazione di test meccanici sui tessuti
	\end{itemize}}

\cventry{1999}{Ingegnere di Qualit\`a di Processo}{Magneti Marelli}{Corbetta}{impiegato}{
Analisi dei processi di produzione e montaggio, controllo difettosit\`a dei componenti
prodotti ed acquisiti per alcune linee della divisione \emph{Quadri di Bordo}. \\}


\subsection{Varie}

\cventry{2003}{Consulente}{Successori Cattaneo S.p.A.}{Albese con Cassano}{}{
	Analisi ed ottimizzazione dei parametri di configurazione di:
	\begin{itemize}
		\item telai a proiettile per la produzione di tessuti serici con l'obiettivo di ridurre le difettosit\`{a} causate dagli arresti
		\item una macchina per l'ispezione visiva automatizzata dei tessuti
	\end{itemize}
	}
\cventry{2003}{Consulente}{Microsystems srl.}{Milano}{}{
\begin{itemize}
	\item progettazione e messa in funzione di una macchina per la produzione di oggetti in cera
	\item progettazione della meccanica di un robot per sorveglianza e telemedicina
\end{itemize}}
\cventry{2003}{Collaboratore}{Raff Progetti srl.}{Galliate}{}{
Progettazione, modellazione numerica e simulazione di sistemi e componenti, in dettaglio:
\begin{itemize}
	\item dimensionamento di un riduttore epicicloidale 
	\item analisi FEM di una reggiatrice
	\item simulazione e dimensionamento di impianti antincendio
	\item progettazione di un sistema di scarico a vuoto
\end{itemize} }

\newpage

%----------------------------------------------------------------------------------------
%	EDUCATION SECTION
%----------------------------------------------------------------------------------------

\section{Formazione}

\cventry{2009--2013}{Laurea magistrale in Ingegneria Informatica}{Politecnico}{Milano}{\textit{110}}{}  % Arguments not required can be left empty
\cventry{2005--2007}{Laurea triennale in Ingegneria Informatica}{Politecnico}{Milano}{\textit{110 cum laude}}{Laurea online: \href{url}{www.laureaonline.polimi.it}}
\cventry{1992--1998}{Laurea in Ingegneria Meccanica}{Politecnico}{Milano}{\textit{100/100}}{Laurea quinquennale}
\cventry{1987--1992}{diploma di Liceo Classico}{Liceo Ginnasio D.Crespi}{Busto Arsizio}{\textit{60/60}}{}

\section{Tesi}

\cvitem{Data}{23/07/2013}
\cvitem{Titolo}{\emph{Mitosis detection in histological images: Algorithms based on machine learning and their performance compared to humans}}
\cvitem{Supervisori}{Professor Vincenzo Caglioti \& ing. Alessandro Giusti}
\cvitem{Descrizione}{La tesi confronta le abilit\`a di umani ed algoritmi di \emph{machine learning} di identificare mitosi in immagini istologiche a parit\`a di condizioni}

\cvitem{Data}{25/09/2007}
\cvitem{Titolo}{\emph{Applicazione di metodologie di apprendimento per rinforzo ad un robot mobile a pendolo inverso}}
\cvitem{Descrizione}{L'elaborato descrive applicazione del metodo di
	apprendimento \emph{NFQ} per consentire ad un robot a pendolo inverso di trovare e mantenere l'equilibrio in modo rapido ed efficiente.}

\cvitem{Data}{08/06/1998}
\cvitem{Titolo}{\emph{Simulazione del disallineamento nei giunti rigidi dei rotori}}
\cvitem{Supervisore}{Professor Nicol\`o Bachschmid}
\cvitem{Descrizione}{La tesi confronta i risultati sperimentali ricavati da un rotore campione con un modello \emph{FEM} sviluppato per descrivere il disallineamento.}


%----------------------------------------------------------------------------------------
%	COMPUTER SKILLS SECTION
%----------------------------------------------------------------------------------------

\section{Conoscenze Informatiche}

\subsection{Linguaggi}

\cvitem{Avanzato}{\textsc{python}, \textsc{C/C++}}
\cvitem{Intermedio}{\LaTeX, \textsc{java}, OpenModelica, \textsc{SQL} }
\cvitem{Base}{\textsc{cuda}, \textsc{html}, Ruby, \emph{R}}

\subsection{Software e Sistemi}

\cvitem{Avanzato}{Linux, \emph{Code Saturne}, \emph{Code Aster}, \textsc{Matlab} }
\cvitem{Intermedio}{OpenOffice, \textsc{Creo Elements}, \textsc{Abaqus}, \textsc{OpenFOAM} }
\cvitem{Base}{\textsc{SolidWorks}, \textsc{SolidEdge} }

\subsection{Hardware}

\cvitem{piattaforme}{\textsc{ARM Cortex M3-M4}, Arduino}

%----------------------------------------------------------------------------------------
%	LANGUAGES SECTION
%----------------------------------------------------------------------------------------

\section{Conoscenze Linguistiche}

\cvitemwithcomment{Italiano}{madrelingua}{}
\cvitemwithcomment{Inglese}{C1 - livello avanzato}{}
\cvitemwithcomment{Tedesco}{A2 - livello elementare}{}


%----------------------------------------------------------------------------------------
%	AWARDS SECTION
%----------------------------------------------------------------------------------------

\section{Riconoscimenti}

\cvitem{2015}{Completamento dei corsi base e avanzato su \textsc{OpenFOAM} presso \href{url}{www.technicalcourses.net} }
\cvitem{2014}{Pubblicazione dell'articolo \emph{"A Comparison of Algorithms and Humans for Mitosis Detection"} nei Proceedings of International Symposium on Biomedical Imaging (ISBI)}
\cvitem{2006}{Completamento del primo livello di corso di \emph{Project Management} presso Siemens }

%----------------------------------------------------------------------------------------
%	INTERESTS SECTION
%----------------------------------------------------------------------------------------

\section{Interessi}

\renewcommand{\listitemsymbol}{-~} % Changes the symbol used for lists

\cvlistdoubleitem{Basket}{Crossfit}
\cvlistdoubleitem{Enigmistica}{MOOCs}
\cvlistitem{3D Printing}

%----------------------------------------------------------------------------------------
%	COVER LETTER
%----------------------------------------------------------------------------------------

% To remove the cover letter, comment out this entire block

\iffalse

\clearpage

\recipient{HR Department}{Corporation\\123 Pleasant Lane\\12345 City, State} % Letter recipient
\date{\today} % Letter date
\opening{Dear Sir or Madam,} % Opening greeting
\closing{Sincerely yours,} % Closing phrase
\enclosure[Attached]{curriculum vit\ae{}} % List of enclosed documents

\makelettertitle % Print letter title

\lipsum[1-3] % Dummy text

\makeletterclosing % Print letter signature


\fi

%----------------------------------------------------------------------------------------

\end{document}